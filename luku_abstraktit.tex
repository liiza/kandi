% Tiivistelmät tehdään viimeiseksi. 
%
% Tiivistelmä kirjoitetaan käytetyllä kielellä (JOKO suomi TAI ruotsi)
% ja HALUTESSASI myös samansisältöisenä englanniksi.
%
% Avainsanojen lista pitää merkitä main.tex-tiedoston kohtaan \KEYWORDS.

\begin{fiabstract}
Tämä kandinaatintyö käsittelee eleentunnistusta Kinect-sensorilla.
Työssä esitellään yleisesti eleentunnistusongelmaa ja miten Kinectin 3D-videokuva vastaa eleentunnistuksen haasteisiin.
Työssä tutustutaan ChaLearn Gesture Challenge -kilpailun kilpailutöihin ja sitä kautta alan uusimpaan tutkimukseen.

Työssä havaitaan, että eleentunnistuksen suurimpia haasteita on luokkien sisäinen varianssi. Sama ele näyttää erilaiselta
esittäjästä ja tilanteesta riippuen. Syvyyskuva kuitenkin pienentää tekstuureista ja värityksestä johtuvaa varianssia.
Syvyyskuvan avulla voidaan luotettavasti erottaa myös ihmishahmotaustasta ja tunnistaa syvyyssuunnassa tapahtuvia liikkeitä.

3D-kuvan luokittelussa voidaan käyttää pitkälti samoja menetelmiä kuin 2D-kuvan luokittelussa. ChaLearn Gesture Challenge -kilpailussa
suurin osa menestyneistä töistä hyödynisikin menetelmiä 2D-videokuvan, puheen tai pelkän kuvan tunnistuksessa. Suosituin menetelmä
menestyneiden töiden joukussa oli HOG/HOF-piirteet yhdistettynä Markovin piilomalliin.  Toinen 
näkökulma kilpailussa oli eleen esittäminen summa- tai liikekuvien avulla. Tällöin ohitetaan videon ajallinen rakenne ja tiivistetään
koko ele yhteen kuva. Tällöinkin luokittelussa hyödynnettiin HOG/HOF-piirteitä tai muita vastaavia kuvan intensiteettivaihteluun perustuvia piirteitä.

Johtopäätöksiä ovat, että syvyyskamera on jo itsessään helpottanut eleentunnistusta ja vanhoja menetelmiä voidaan hyödyntää
pienin muutoksin myös 3D-kuvaan. Kehittettävää löytyy kuitenkin vielä erityisesti yleistettävyydessä ja opetusdatan määrän vähentämisessä.
%
%Tiivistelmätekstiä tähän (\languagename). Huomaa, että tiivistelmä tehdään %vasta kun koko työ on muuten kirjoitettu.
\end{fiabstract}

%\begin{svabstract}
%  Ett abstrakt hit 
%%(\languagename)
%\end{svabstract}

%\begin{enabstract}
% Here goes the abstract 
%%(\languagename)
%\end{enabstract}


